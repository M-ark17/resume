\documentclass[10pt]{article}

%-------------------------------------------------------------------------
%\usepackage{fullpage}
\usepackage{multicol}
\usepackage{amssymb}
\usepackage[a4paper]{geometry}
\usepackage{color}
\usepackage{multirow}
\usepackage{bm}
\usepackage{array}
\usepackage{amsmath,latexsym} 
\usepackage{csquotes}
\usepackage{verbatim}
\usepackage{ tipa }
\usepackage{enumitem}
%\usepackage[autostyle]{csquotes}

\usepackage[pdftex]{graphicx}
\pagestyle{empty}
%\definecolor{bl}{RGB}{173,216,230} %ankit   123, same %pratik
%\definecolor{bl}{rgb}{0.2, 0.9, 0.9} %light green
%\definecolor{bl}{RGB}{133,220,230} % pratik ********
%\definecolor{bl}{RGB}{173,210,240}  % new


\definecolor{bl}{RGB}{173,216,230}  %final
%\definecolor{bl}{RGB}{166,226,235}  %pratik
\definecolor{mygray}{gray}{0.44}


\newcommand{\hilight}[1]{\colorbox{light-gray}{#1}}
\geometry{left=0.65in, right=0.57in, top=0.70in, bottom=0.5in}

%-------------------------------------------------------------------------
%font san serif family
%\upshape
%\mdseries
\fontfamily{pcr}  % ld
%%\fontfamily{SansSerif}

%-------------------------------------------------------------------------
\setlength{\parindent}{-2mm}
\usepackage{enumitem}
\newlist{myitemize}{itemize}{2}
\setlist[myitemize,1]{label=$\bullet$,leftmargin=8mm,topsep=2mm,itemsep=1.1mm,itemindent=0mm,parsep=-0.5mm}
\setlist[myitemize,2]{label=$-$,leftmargin=9mm,topsep=1mm,itemsep=1mm,itemindent=0mm,parsep=-1mm}
\newcommand\hs{1.3mm}		%horizontal space
%-------------------------------------------------------------------------
\setlength{\tabcolsep}{5pt}
%-------------------------------------------------------------------------

%-------------------------------------------------------------------------
%for horizontal lines
\usepackage[pdftex]{graphicx}
\newcommand{\HRule}{\rule{\linewidth}{0.2mm}}
%-------------------------------------------------------------------------

%-------------------------------------------------------------------------
%spaces in itemize
%\usepackage{enumitem}
%-------------------------------------------------------------------------


\begin{document}

%--------------------------------------------------------------------------

\vspace*{137pt}    % EARLIER   *********************


%--------------------------------------------------------------------------
%-----------------------------------------------------------------
\colorbox{bl}{\makebox[\textwidth][l]{\bfseries \color{black} AREAS OF INTEREST}}
\begin{itemize}[itemindent=-4.8mm]
		\vspace{-0.15cm}
	\item[] \hspace{0.01cm}Wireless Communication, Digital Signal Processing, Image Processing.
	\vspace{-0.1cm}
\end{itemize}
%------------------------------------------------------------------------

%-----------------------------------------------------------------------
\colorbox{bl}{\makebox[\textwidth][l]{\bfseries \color{black} MAJOR PROJECT AND SEMINAR}}% \\
\vspace{-0.1cm}
\begin{itemize}[leftmargin=0.4cm]
\item \textbf{M.Tech Project: An efficient channel estimation scheme in MIMO TDD systems}
\hfill{(\textit{May'19 - Present})} \\
Guide: \textit{Prof. Kumar Appaiah, Electrical Engineering, IIT Bombay}

\vspace{-0.1cm}
%\begin{itemize}
  \textbf{Objective:} To design an \textbf{efficient channel estimation} scheme in TDD with the help of \textbf{feedback} in MIMO Communication which will reduce the effect of pilot contamination on MIMO channel estimation.\\
%\end{itemize}
%\vspace{-0.2cm}
%\vspace{-0.25cm}
\textbf{Completed work:} Implemented \textbf{Multi Cell MMSE based MIMO precoding} in mulitiple antennas cellular systems which used non-orthogonal pilots for channel estimation.
\vspace{-0.25cm}
\begin{itemize}

\item Analysed and implemented \textbf{covariance based channel estimation} which uses Bayesian Estimation in single cell multi antenna system and observed its performance based on rate vs number of antennas.
\vspace{-0.15cm}
\item Implemented \textbf{Kalman estimation} for multi antenna cellular system.  
\vspace{-0.1cm}
\item Parameterized the feedback for \textbf{postcoder} in massive \textbf{MIMO TDD} systems with multi antenna users. 
\vspace{-0.1cm}
\item Formulated a lower bound on the achievable rate for systems with \textbf{perfect CSIT} and \textbf{partial CSIR}.
\vspace{-0.1cm}
\end{itemize}
\vspace{-0.2cm}

%\vspace{-0.2cm}
\textbf{Ongoing work:} Working on implemention a basic precoder which utilises the information obtained from \textbf{Kalman} estimate in \textbf{coordinated MIMO} systems.

\vspace{-0.1cm}
\item \textbf{M.Tech Seminar: Inter cell interference in Multi Cell MIMO systems 
} 
\hfill{(\textit{Jul'18 - Nov'18})}\\
Guide: \textit{Prof. Kumar Appaiah, Electrical Engineering, IIT Bombay}\\\vspace{-0.68cm}
	\begin{itemize}
	\item Studied the structure and working of \textbf{MIMO systems} and impact on BER on using Non-orthogonal pilot sequences for channel estimate.\vspace{-0.1cm}
	\item Simulated \textbf{BER vs SNR} for MIMO systems in Interference and Interference free scenarios to study the impact of pilot contamination on the performance of the system.\vspace{-0.1cm}

	\end{itemize}\
\vspace{-0.8cm}	

\end{itemize}
%------------------------------------------------------------
\vspace{0.2cm}
\hspace{-0.25cm}
\colorbox{bl}{\makebox[\textwidth][l]{\bfseries \color{black}  WORK EXPERIENCE \& NON ACADEMIC PROJECT}}

\vspace{-0.1cm}
\begin{itemize}[leftmargin=0.4cm]
	\item \textbf{Interview Management Software \textpipe  \hspace{0.05cm} Electrical Engineering Department, IITB} \hfill{(\textit{Feb'18 - present})}\\
	Guide: \textit{Prof. Bikash Kumar Dey, Prof. Madhu N. Belur, Electrical Engineering, IIT Bombay}\\
\vspace{-0.65cm}
\begin{itemize}

		\item \textbf{Lead role }in building an online system that made \textbf{automatic
coordination }across interview committees possible through their interface.\vspace{-0.1cm}
		\item Online system allowed committees to decide in \textbf{real-time} using a \textbf{cross platform web application} about interviews.
		\item Built various other \textbf{peripheral
interfaces} to collect data at different times from students and other sources.
		\item This system was used \textbf{successfully} in the last \textbf{3} admission sessions.
	\end{itemize}
\end{itemize}
\vspace{-0.55cm}
\begin{itemize}[leftmargin=0.4cm]
	\item \textbf{Systems Engineer \textpipe  \hspace{0.05cm} Infosys Technology Ltd} \hfill{(\textit{Dec'15 - Jul'17})}\\
	\vspace{-0.65cm}
	\begin{itemize}
	    \item  \textbf{Tools Used:} Oracle Peoplesoft.\vspace{-0.07cm}
		\item \textbf{Roles and responsibilities:} Part of the team which developed an application which automates billing for the customers of the client.\vspace{-0.1cm}
		\item Wrote SQL queries to fetch data to the module and to develop features using Oracle Peoplesoft ERM tool.
		\item Assisted in designing billing template in XML and in completion of Technical Document Report for the project.
	
	\end{itemize}
\end{itemize}

%----------------------------------------------------------------------

\colorbox{bl}{\makebox[\textwidth][l]{\bfseries \color{black} KEY COURSE PROJECTS}}
\vspace{-0.5cm}
\begin{itemize}[leftmargin=0.4cm]


	
\item \textbf{Scheduling in 4G LTE }
\hfill{(\textit{Jan'18 - Apr'18})}\\
Guide: \textit{Prof. Abhay Karandikar, Electrical Engineering, IIT Bombay}\\
\vspace{-0.7cm}
	\begin{itemize}
	\item Studied about different scheduling schemes for \textbf{resource block allocation} to users in \textbf{LTE} systems.\vspace{-0.1cm}
	\item Implemented channel aware scheduling schemes such as Maximum Throughput, Proportional Fairness, Throughput to Average and compared all three scheduling schemes based on metrics such as \textbf{cell throughput, average user throughput and Jain Fairness index}.
    \end{itemize}

	\vspace{-0.25cm}
	
	
\item \textbf{Simulation of Cellular System in Octave}
\hfill{(\textit{Jan'18 - Apr'18})}\\
Guide: \textit{Prof. Abhay Karandikar, Electrical Engineering, IIT Bombay}\\\vspace{-0.7cm}
	\begin{itemize}
	\item Computed \textbf{SIR}, \textbf{blocking probability} for different cluster sizes and \textbf{sectoring}.\vspace{-0.1cm}
	\item Analyzed \textbf{handover} process and \textbf{ping-pong} rate for different user mobilities and hysteresis values.\vspace{-0.1cm}
	\item Analyzed BER performance for \textbf{space} and \textbf{time diversity} in a slow flat fading Rayleigh channel.\vspace{-0.1cm}
	\item Analyzed BER performance for a single-cell and multi-cell scenario in a \textbf{CDMA} cellular system.
    \end{itemize}

	\vspace{-0.25cm}	
	


%\newpage
\item \textbf{Image Dehazing using color attenuation prior and dark channel prior 
} \hfill{(\textit{Jul'18 - Nov'18})}\\
Guide: \textit{Prof. Amit Sethi, Electrical Engineering, IIT Bombay}\\\vspace{-0.68cm}
	\begin{itemize}
	\item Implemented \textbf{Color Attenuation Prior and Dark Channel Prior} techniques to estimate the Depth map.\vspace{-0.1cm}
	\item Implemented \textbf{Guided Filter} to reconstruct the Haze-free image using Hazy Image and its Depth map.\vspace{-0.1cm}
	\end{itemize}\
\vspace{-0.75cm}	

\item \textbf{Basic Image Editor tool in Python 
} 
\hfill{(\textit{Jul'18 - Nov'18})}\\
Guide: \textit{Prof. Amit Sethi, Electrical Engineering, IIT Bombay}\\\vspace{-0.68cm}
	\begin{itemize}
	\item Built a \textbf{GUI tool using pyQt} to implement Histogram Equalisation, Gamma correction, log transformation, Horizontal and Vertical edge detection using Sobel operators, blurring and sharpening with a mechanism to control the extent of blurring and sharpening respectively .\vspace{-0.1cm}
	\item Implemented \textbf{Image Deblurring} using Inverse filter, Truncated inverse filter, Weiner filter, Constrained least square filter and analysed the performance with help of metrics PSNR and SSIM.\vspace{-0.1cm}
	\end{itemize}\
\vspace{-0.75cm}

\item \textbf{Wavelet based leaders and P leaders in Multi
Fractal Analysis} 
\hfill{(\textit{Jul'18 - Nov'18})}\\
Guide: \textit{Prof. Vikram M Gadre, Electrical Engineering, IIT Bombay}\\\vspace{-0.68cm}
	\begin{itemize}
	\item Studied about \textbf{p-exponents} and \textbf{p-leaders} which measure negative regularity which appear in most real time signal analysis.\vspace{-0.1cm}
	\item Simulated p-leaders for several signals and were able to prove their convergence with \textbf{DWT based Wavelet leaders} as \textbf{p} becomes large.
\vspace{-0.1cm}

	\end{itemize}\
	\vspace{-0.75cm}

\item \textbf{Spam URL classification using Machine Learning}
\hfill{(\textit{Jan'19 - Apr'19})}\\
Guide: \textit{Prof. Gaurav S kasbekar, Electrical Engineering, IIT Bombay}\\
\vspace{-0.7cm}
	\begin{itemize}
	\item Studied 3 among the Top-10 vulnerabilities of \textbf{OWASP} Standard mainly \textbf{XML external entity} attack, \textbf{SQL injection}, \textbf{cross site scripting} with practical implementation and proposed solutions.\vspace{-0.1cm}
	\item Spam URL classification using Machine Learning Techniques like \textbf{Logistic Regression, Naive Bayes, Support Vector Machine, One-vs-Rest}.\vspace{-0.1cm}
	\item An increase of more than \textbf{2 percent} in accuracy was obtained by replacing logistic regression by one vs rest classification.\vspace{-0.1cm}
    \end{itemize}

	\vspace{0.1cm}


\end{itemize}
%-------------------------------------------------------------------------------------------
\hspace{-0.17cm}\colorbox{bl}{\makebox[\textwidth][l]{\bfseries \color{black} RELEVANT COURSES}}% \\
	\vspace{0.1cm} 


 	
		\begin{tabular}{ l l l }
		\hspace{0.55cm}\textbullet\ Wireless and Mobile Communications &  \textbullet\ DSP and its Applications & \textbullet\ Image Processing \\
		\hspace{0.55cm}\textbullet\ Apllied Linear Algebra & \textbullet\ Statistical Signal Analysis & \textbullet\ Network Security \\
		\hspace{0.55cm}\textbullet\ Information Theory and Coding & \textbullet\ Optimisation & \textbullet\ Wavelets\\
	\end{tabular}
	
	
	
	\vspace{0.1cm}
%----------------------------------------------------------------------------------
\colorbox{bl}{\makebox[\textwidth][l]{\bfseries \color{black} TECHNICAL SKILLS}}\\
\begin{tabular}{m{1in}m{0.20in}m{4.5in}}
	\\[-3mm]
	\hspace{\hs} \hspace{0.12cm}\textbf{\textbf{Languages}} &: & {{C, C++, Python, Bash scripting, HTML, PHP}} \\
	\\[-3.5mm]
	\hspace{\hs} \hspace{0.12cm}\textbf{\textbf{Tools}} &: & {Matlab/Octave, \LaTeX, Git.}\\
	\\[-4mm]
\end{tabular}\\



%-------------------------------------------------------------------------------------------
\colorbox{bl}{\makebox[\textwidth][l]{\bfseries \color{black} POSITIONS OF RESPONSIBILITY}}

\begin{itemize}[leftmargin=0.4cm]
\vspace{-0.1cm}
\item \textbf {System Administrator: PC Lab, Electrical Department, IIT Bombay} 
\hfill{(\textit{Jul'17 - Present})}\\[-0.6cm]
    \begin{itemize}
	\item Building and maintaining website of EE department, maintaining TA feedback and allotment portals.\vspace{-0.1cm}
	\item Provide mail service, storage space, computing facilities and network facilities to department.\vspace{-0.1cm}
	\item Designed online portals and automated Interviews co-ordination in the department admission process.
	\end{itemize}
	\vspace{-0.25cm}
	
	
\item \textbf{Web Nominee: Post Graduate Academic Council, IIT Bombay}
\hfill{(\textit{Jul'18 - Jun'19})}\\[-0.6cm]
    \begin{itemize}
			\item Designed new web portal for PGAC which is used by all the Post Graduate students of the institute. \vspace{-0.1cm}	
	\end{itemize}
	\vspace{-0.25cm}
\begin{comment}
\item \textbf{Mess Secretary: Hostel-1, IIT Bombay}
\hfill{(\textit{Sep'17 - Mar'18})}\\[-0.6cm]
    \begin{itemize}
            \item Managed all mess related activites for a mess which catered for 250+ students with an approximate budget of 8,00,000/- per month.\vspace{-0.1cm}
            %\item Responsible for verification of mess bills, coordinating with vendors.
			%\item Reduced mess bill from Rs-140/- per student per day to around Rs-110/- per students per day in span of two months.\vspace{-0.1cm}
		\end{itemize}
		\vspace{-0.25cm}	
\end{comment}

    	
\end{itemize}
\vspace{0.15cm}

%---------------------------------------------------------------------------------
\colorbox{bl}{\makebox[\textwidth][l]{\bfseries \color{black} CO \& EXTRA CURRICULAR ACTIVITIES}}
\vspace{-0.45cm}
\begin{itemize}[leftmargin=0.4cm]
		\item {Completed \textbf{Machine Learning} course by \textbf{Andrew Ng} from Coursera. }\hfill{(\textit{2019})}\vspace{-0.2cm}
	\item {Conducted a introductory session on \textbf{Linux, vim and Git} as a part of Bridge Course which helps in smooth tranisition of new joinees to institute. }\hfill{(\textit{2019})}\vspace{-0.2cm}
	\item {Voluntered for an introductory session on \textbf{Python} which was conducted as a part of Bridge Course.}
	\hfill{(\textit{2019})}\vspace{-0.2cm}
    \item {Completed a \textbf{100 hrs} course on \textbf{Chinese} language conducted by International Relations office IIT Bombay. }\hfill{(\textit{2019})}\vspace{-0.2cm}
    \item {Completed Basic course in \textbf{French} Language from Vivekananda Institute of Languages, Hyderabad.}\hfill{(\textit{2013})}\vspace{-0.2cm}

    \item {Completed Diploma in spoken English from Vivekananda Institute of Languages, Hyderabad.} \hfill{(\textit{2012})}\vspace{-0.2cm}
   \item {Active member of \textbf{National Service Scheme (NSS)} for 2 years and attended a camp conducted in a village to perform social activites like conducting medical camps, cleaning and painting common facilites like village panchayat, temple etc.}\hfill{(\textit{2013})}\vspace{-0.2cm}
   \item {Voluntered for \textbf{$1^{st}$ World Parliament on Spirituality} for a week as part of NSS activity.}\hfill{(\textit{2012})}\vspace{-0.2cm}
    
\end{itemize}

%-------------------------------------------------------------------------------------------------



\end{document}
